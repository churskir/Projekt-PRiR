\documentclass[a4paper]{article}

\usepackage{amsmath}
\usepackage[T1]{fontenc}
\usepackage[polish]{babel}
\usepackage[utf8]{inputenc}
\usepackage{lmodern}
\usepackage{graphicx}
\usepackage{geometry}
\usepackage{float}
\selectlanguage{polish}

\geometry{
	a4paper,
	total={170mm,257mm},
	left=20mm,
	top=20mm,
}

\title{Programowanie Równoległe i Rozproszone\\Projekt}
\author{Radosław Churski \\ Paweł Kurbiel}

\begin{document}
\begin{titlepage}
\maketitle
\end{titlepage}
\tableofcontents
\section{Temat projektu}
Jako temat projektu wybraliśmy model symulacyjny prostego ruchu drogowego. Można w nim rozróżnić następujące podmioty:
\begin{description}
\item[samochód] porusza się po drogach i skrzyżowaniach. Jeśli napotka przed sobą skrzyżowanie może na nie wjechać tylko wtedy, kiedy odpowiedni sygnalizator świetlny ma zapalone zielone światło. Samochód zna tylko najbliższe otoczenie.
\item[skrzyżowanie] jest wyposażone w dwa sygnalizatory świetlne, po jednym dla każdej z dwóch osi. Z założenia, samochody jadące na przeciw siebie "mają" to samo światło. Skrzyżowanie odpowiada za sterowanie sygnalizacją świetlną.
\item[sygnalizacja świetlna] może mieć jeden z dwóch kolorów, zielony lub czerwony.
\item[mapa] reprezentuje świat w którym poruszają się samochody i znajdują się drogi oraz skrzyżowania. Z mapy samochody dowiadują się o swoim najbliższym otoczeniu.
\end{description}
\section{Realizacja}
\subsection{Uruchomienie; dane wejściowe}
Program przyjmuje trzy argumenty:
\begin{itemize}
\item ścieżka do pliku z mapą
\item liczba samochodów
\item czas pracy programu
\end{itemize}
\subsubsection{Plik z mapą}
\subsection{Implementacja samochodów}
\subsection{Implementacja skrzyżowań ze światłami}
\subsection{Wyjście}
\section{Podsumowanie}
\end{document}